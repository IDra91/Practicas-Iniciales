\documentclass[landscape,12pt]{report} 
\usepackage[spanish]{babel}
\usepackage{graphicx}
\usepackage{eso-pic}
\usepackage{hyperref}
\usepackage{multicol}
\usepackage{anysize}
\marginsize{1.5cm}{1.5cm}{1.0cm}{1.0cm}

\pagestyle{empty}
\begin{document}

	\begin{tabular}{c c c}
			%------------------------------------%
			%-------------Columna No. 1----------%
			%------------------------------------%
			\begin{minipage}[t]{8.2cm} 
				\begin{center}
		\begin{center}
					
				   		
					 	\begin{flushleft}
					 	\Large
					 	Grupo 12
					 	
					 		\medskip
					 	\it Integrantes:
					 	
					 		\medskip
					 	 	\normalsize 
					 	 	
					 	 	
					 	 	Jerem\'ias Alberto Ortega Fonseca. 201212766
					 	 	
					 	 	Abel Guti\'errez De Leon.  201212788
					 	 	
					 	 	Carlos David Ramirez Altan. 201213132
					 	 	
					 	 	William Efrain Veliz Arteaga. 201212902
					 	 	
					 	 	Manuel Gonzalo Rivera Estrada. 201212747
					 	 	
					 	 	\end{flushleft}
					 	 	
					 	 	\medskip
					 	
				   		\medskip
				   	 	
				   	 	\medskip
				   	 	
				   	 	\medskip
				   	 	
				   	 	\medskip
				   	 	
				   	 	\medskip
				   	 	
				   	 	\medskip 	
						   	
                    \LARGE{\textbf{MANTENIMIENTO DE UNA COMPUTADORA COMPAQ PRESARIO R3000}}
				   	 	\medskip
				   	 	
				   	 	\medskip
					 	\begin{center}
							\includegraphics[width=1.0\textwidth]{./Grupo}
						\end{center}
					
					 	\medskip
					 	
					 	\title{
				   	 \date{\today}
					 }
						
					\end{center}		
				
				
				\medskip
					
				\end{center}
					 
					
				
					\medskip
					-Realizar la limpieza en una Compaq Presario R3000.
					
					\medskip
					\begin{flushright}
				    	\includegraphics[width=0.8\textwidth]{./grupo12}
			   	\end{flushright}
			    	
					\medskip
					-Mantener el equipo desenchufado todo el tiempo.
					
					\begin{center}
				    	\includegraphics[width=0.7\textwidth]{./desen}
			   	\end{center}

					
			\end{minipage}& 
			%------------------------------------%
			%-------------Columna No. 2----------%
			%------------------------------------%
			\begin{minipage}[t]{8.2cm}%comienza la segunda pagina
			 
			\vspace{0.7cm}
            	\begin{center}
				\Large \bf
				Recomendaciones para Realizar el mantenimiento 
				\end{center}

				\medskip
					-Evitar tener contacto con las partes de la tarjeta madre, sin haberse quitado la energ\'ia electrost\'atica antes, esto puede hacerse pasando las manos por alguna superficie, como el suelo o la pared.
					
				
					
					\medskip
					-Antes de desconectar o interactuar con cualquier componente, asegurarse primero de quitar la bater\'ia para evitar que est\'e presente energ\'ia en ellos.

					
			   	
			   		\medskip

					 No fuerce los componentes de la m\'aquina. Tr\'atelos con cuidado y lleve a 								cabo todas las precauciones.
			   	
			   	\medskip
					-Siempre ir guardando los tornillos que se van quitando, en un lugar espec\'ifico, organizadamente para tener una idea de d\'onde va cada tornillo.
			   	
			   	\medskip
			   	    -Utilizar el desarmador en un \'angulo de 90 grados, y aplicar una fuerza horizontal para evitar que el tornillo sea afectado.
				\end{minipage}
				& 
			%------------------------------------%
			%-------------Columna No. 3----------%
			%------------------------------------%
				\begin{minipage}[t]{8.2cm} %comienza la tercera pagina
					
					\vspace{0.7cm}
					%Portada del trifoliar 
						\begin{center}
				\Large \bf
				Recomendaciones para el cuidado 
				\end{center}
					
					
					
					\medskip
					    -Para que la computadora no sea afectada por ralentizaci\'on o por descomposici\'on parcial o total de la computadora, se recomienda que el mantenimiento se le de al menos una vez cada tres meses.
					    
					    
					    
					    \medskip
					    -Siempre evitar colocar la computadora en un lugar que sea propenso a recibir polvo de manera excesiva.
					    
					    
					    
					    \medskip
					    -El uso del limpia contactos debe ser solo en caso de que se sospeche que un contacto deje de estar en un funcionamiento \'optimo.
		
		\begin{center}
				    	\includegraphics[width=0.8\textwidth]{./plg}
			    	\end{center}					
				
				\end{minipage}

\end{tabular}
								
								
		  %-------------------------------------------------------------%
		  %-------------------Segunda P\'agina del trifoliar--------------%
		  %-------------------------------------------------------------%
\newpage %nueva p�gina

\begin{tabular}{c c c}
			%------------------------------------%
			%-------------Columna No. 1----------%
			%------------------------------------%
			\begin{minipage}[t]{8.2cm}
				
				\begin{center}
				\Large \bf
				\title{\textbf{Mantenimiento de una Computadora Compaq Presario R3000}} 
				\end{center}
				
				El mantenimiento de una computadora es el procedimiento indispensable para prevenir o corregir fallas que puedan estar presentes en el rendimiento de una computadora

\medskip 
El per\'iodo de mantenimiento depende de m\'ultiples factores, de los cuales se puede mencionar: el ambiente alrededor de la computadora (polvo, aire, temperatura, etc.), si es un equipo nuevo o ya se encuentra muy usado y la cantidad de horas diarias que opera la misma.

\medskip
Las herramientas utilizadas para el mantenimiento de una computadora son: 

\medskip
-Un desarmador (plano o de cruz, seg\'un sea el tipo de tornillos que tenga la PC). 

-Espuma limpiadora

-Una brocha

-Un trapo

-Aire comprimido

-Limpiador de contactos

\begin{center}
    \includegraphics[width=0.8\textwidth]{./tools}
    \end{center}

\medskip

\medskip
					
				
				
				
			\end{minipage}& 
			%------------------------------------%
			%-------------Columna No. 2----------%
			%------------------------------------%
			\begin{minipage}[t]{8.2cm}				
		
					\begin{center}
					\large
					\bf
					\it
					Pasos para el mantenimiento
					\end{center}
				
				\medskip	
				1. Se apaga y se desconecta el equipo.
				
				
				\medskip	
				2. Se procede a agarrar la espuma limpiadora y rociarla sobre la computadora, para luego pasar el trapo y pasarlo por todo el equipo, (tambi\'en se puede rociar directamente la espuma sobre el trapo y pasarlo). 
				
				\begin{flushleft}
					\includegraphics[width=0.8\textwidth]{./espuma}
				\end{flushleft}
				
				\medskip
					3. Se comienza a desatornillar la parte inferior de la computadora.
					\begin{center}
				  	\includegraphics[width=0.8\textwidth]{./tornillos}
			  	\end{center}
			  	
			  	\medskip
					 4. Se comienza a quitar uno por uno los componentes de la computadora, siempre procurando quitar la bater\'ia primero, para procurar que no se vea afectada la computadora,  una vez revisado que no falten tornillos por ning\'un lado, se procede a pasar la brocha sobre los ventiladores.
					 
					 							
					 
			    	
				\end{minipage} & 
				
			%------------------------------------%
			%-------------Columna No. 3----------%
			%------------------------------------%
				\begin{minipage}[t]{8.2cm}
				
					
				
			    	\medskip
			    	5. Para los lugares dif\'iciles, por donde no llegue a pasar la brocha, se procede a utilizar el aire comprimido para eliminar todo rastro de polvo que sobre. 
			    	 \begin{center}
				    	\includegraphics[width=0.8\textwidth]{./aire}
			    	\end{center}
			    	\medskip
					 6. Se procede a rociar limpia contactos, sobre los conectores donde se presente alg\'un tipo de suciedad para evitar problemas en los conectores.
				
					  \begin{center}
				    	\includegraphics[width=0.8\textwidth]{./limpia}
			    	\end{center}
			    	
			    	
			    	\medskip
			    7. Finalmente, se procede a atornillar los componentes, luego cerrar y atornillar la tapadera del equipo.
				
			    	
				
			 
			  \end{minipage}
			  						
	
			
\end{tabular}
		 
\end{document}
